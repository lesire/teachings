\documentclass[a4paper]{article}
\usepackage[francais]{babel}
\usepackage[T1]{fontenc}
\usepackage[utf8]{inputenc}
\usepackage[figbotcap]{subfigure}  % use for side-by-side figures - ver  http://linuxdicas.wikispaces.com/latex
\usepackage{graphicx,amssymb,algorithmic,algorithm,url,hyperref,tikz,exam}
\usepackage[babel=true,kerning=true]{microtype} % <- bug labels tikz
\usetikzlibrary{arrows,shapes,snakes,automata,backgrounds,petri}

\graphicspath{{../figures/}}

\paper{IN310}
\school{ISAE - Supaero}
\subject{EXAMEN}
\title{Modèles de Systèmes Embarqués : modèles discrets, modèles hybrides}
\semester{3A SEM}
\year{2010-2011}
\note{Tous documents autorisés.}

\begin{document}

\begin{center}
  \large{\bf \textsc{Exercice} A\\[5mm]
    \textsc{Réseaux de Petri}}
\end{center}

On modélise par le réseau de Petri de la figure \ref{petri} le fonctionnement d'un atelier qui doit réaliser deux types de pièces.
Pour le premier type de pièce, un opérateur doit placer la pièce dans une machine puis la retirer quand la machine a fini son travail.
Le deuxième type de pièce est traité de façon manuelle par un des deux opérateurs.

\begin{figure}[!h]
\begin{center}
	\begin{tikzpicture}[node distance=1.3cm,>=stealth',bend angle=45,auto]
	\tikzstyle{place}=[circle,thick,draw=blue!75,fill=blue!20,minimum size=6mm]
  	\tikzstyle{transition}=[rectangle,thick,draw=black!75,fill=black!20,minimum size=4mm]

    \node [transition] (t1) [label=above:$t_1$] {};
    \node [place] (p1) [below of=t1,label=left:$p_1$] {};
    \node [transition] (t2) [below of=p1,label=right:$t_2$] {};
    \node [place] (p2) [below of=t2,label=left:$p_2$] {};
    \node [transition] (t3) [below of=p2,label=right:$t_3$] {};
    \node [place] (p3) [below of=t3,label=left:$p_3$] {};
    \node [transition] (t4) [below of=p3,label=left:$t_4$] {};
    \node [place,tokens=1] (p4) [left of=p2,label=left:$p_4$] {};
    \node [place,tokens=2] (p5) [right of=p1,label=left:$p_5$] {};
    \node [place] (p6) [right of=p5,label=right:$p_6$] {};
    \node [transition] (t5) [above of=p6,label=right:$t_5$] {};
    \node [transition] (t6) [below of=p6,label=right:$t_6$] {};

	\draw (t1)
		edge [pre,bend left] (p5)
		edge [post] (p1);
	\draw (t2)
		edge [pre] (p1)
		edge [pre,bend right] (p4)
		edge [post] (p2)
		edge [post,bend right] (p5);
	\draw (t3)
		edge [pre] (p2)
		edge [pre] (p5)
		edge [post] (p3)
		edge [post,bend left] (p4);
	\draw (t4)
		edge [pre] (p3)
		edge [post,bend right] (p5);
	\draw (t5)
		edge [pre,bend right] (p5)
		edge [post] (p6);
	\draw (t6)
		edge [pre] (p6)
		edge [post,bend left] (p5);
	\end{tikzpicture}
	\caption{Réseau de Petri de l'atelier.}
	\label{petri}
\end{center}
\end{figure}

\begin{questions}
\question Donner la signification des six places de la figure \ref{petri}.
\question Donner les matrices Pre et Post du réseau.
\question Etudier les propriétés du réseau :
	\begin{enumerate}
	\item le réseau est-il borné ?
	\item le réseau est-il vivant ?
	\item le réseau est-il ré-initialisable ?
	\end{enumerate}
\question Trouver au moins deux invariants de place du réseau. Que représentent-t-ils ?
\end{questions}

\begin{center}
  \large{\bf \textsc{Exercice} B\\[5mm]
    \textsc{Automates}}
\end{center}

\end{document}

