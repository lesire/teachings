\documentclass[compress]{beamer}

\usetheme{Szeged}
\usepackage[T1]{fontenc}
\usepackage[utf8]{inputenc}
\usepackage[frenchb]{babel}
\usepackage[babel=true,kerning=true]{microtype}
\usepackage{tikz,listings,algorithm,algorithmic,hyperref}
\usetikzlibrary{automata,shapes,snakes,arrows}

\tikzstyle{every picture}=[sibling distance=3cm, shorten >=1pt, node distance=2cm,%
	>=stealth', bend angle=10, auto, initial text=]

\title[SEM IN310 - Réseaux de Petri]{Réseaux de Petri\\IN310 - Modèles des SE}
\author[Charles Lesire]{Charles Lesire-Cabaniols (ONERA / DCSD)\\{\tt charles.lesire@onera.fr}}
\date[2010-2011]{3A-SEM - 2010-2011}

\graphicspath{{../figures/}}
\lstset{basicstyle=\tiny,tabsize=2,
	emph={define,domain,requirements,strips,typing,types,predicates,
		action,parameters,precondition,vars,effect,objects,init,goal},%
	emphstyle=\bf}

\begin{document}

\begin{frame}
\titlepage
\end{frame}

\begin{frame}
\tableofcontents[hidesubsections]
\end{frame}

%%%%%% INTRODUCTION %%%%%%%
\section{Introduction}
\begin{frame}
\tableofcontents[hideothersubsections]
\end{frame}

\subsection{Introduction}
\begin{frame}{Introduction}
\begin{itemize}
\item 1962, Carl Adam Petri: Communication et composition entre automates
\item Outil de modélisation de systèmes dynamiques : permet de raisonner sur les objets, les ressources et leur changement d'état
\item Outil mathématique (formel) et outil graphique
	\begin{itemize}
	\item permet de représenter le vrai parallélisme, la concurrence, contraintes de précédence,
	\item analyse de bonnes propriétés (vivacité, borné, etc.) et propriétés structurelles : aide efficiente durant les phases de conception
	\item peut être simulé et implémenté directement par un joueur de RdP
	\end{itemize} 
\end{itemize} 
\end{frame}

\begin{frame}{Introduction}
\begin{itemize}
\item Applications:
	\begin{itemize}
	\item évaluation de performances, 
	\item analyse et vérification formelles, 
	\item protocoles de communication, 
	\item contrôle de systèmes de production, 
	\item systèmes d'information (organisation d'entreprises), 
	\item gestion de bases de données, 
	\item IHM, etc.
	\end{itemize} 
\end{itemize} 
\end{frame}

\begin{frame}{Introduction}
\begin{itemize}
\item Etat : les différentes {\it phases} par lesquelles passe le système;
\item Variables d'état : ensemble de variables qui permettent de connaître l'état du système.
	\begin{itemize}
	\item Système continu : les variables d'état évoluent continuellement dans le temps;
	\item Système à événements discrets : les variables d'état changent {\it brusquement} à certains instants
	\end{itemize}
\item Evénement : son occurrence fait changer l'état du système
\item Activité : {\it boîte noire} représente l'évolution du système entre 2 événements
\item Processus : séquence d'événements et d'activités \\
	$\leadsto$  coopération, compétition, parallélisme
\end{itemize}
\end{frame}

\subsection{Présentation informelle}     
\begin{frame}{Présentation informelle}
\begin{block}{Éléments de base}
\begin{itemize}
\item \structure{Place :} interprétée comme condition, état partiel, ensemble de ressources
\item \structure{Transition :} associée à un événement qui a lieu dans le système
\item \structure{Jeton :} indique que la condition associée à la place est vérifiée (ou le nombre d'éléments qui la vérifient)
\end{itemize}
\end{block}
\end{frame}

\begin{frame}{Présentation informelle}
\begin{block}{Comportement dynamique}
\begin{columns}
	\begin{column}{.7\textwidth}
	\begin{itemize}
	\item état : répartition des jetons dans les places,
	\item occurrence d'un événement : tir de la transition,
		\begin{itemize}
		\item  enlever les jetons des places d'entrée,
		\item  mettre les jetons dans les places de sortie.
		\end{itemize}
	\end{itemize}
	\end{column}	
	\begin{column}{.3\textwidth}
		\includegraphics[width=3.2cm]{frdp}
	\end{column}
\end{columns}
\end{block}
\end{frame}

%%%%% MODELE FORMEL %%%%%
\section{Modèle formel}
\begin{frame}
\tableofcontents[hideothersubsections]
\end{frame}

\subsection{Définition}
\begin{frame}{Définitions}
\begin{itemize}
\item Modèle formel, peut être caractérisé par :
	\begin{itemize}
	\item graphe avec comportement dynamique ; représentation naturelle pour le concepteur,
	\item ensemble de matrices d'entiers : comportement dynamique décrit par un système linéaire :  représentation naturel pour l'ordinateur ;
	\item système de règles: peut être utilisé avec les techniques d'I.A;
	\end{itemize}     
\item Validation par analyse et simulation ;
\item Représente: parallélisme, synchronisme, séquence, conflit, concurrence.
\end{itemize}
\end{frame}
 
\begin{frame}{Définitions}
\begin{block}{Réseaux de Petri $R = <\!P, T, Pre, Post\!>$}
	\begin{itemize}
	\item $P$ est un ensemble fini de places de dimension $n$;
	\item $T$ est un ensemble fini de transitions de dimension $m$;
	\item $Pre : P \times T \rightarrow \mathbb{N}$ est l'application d'{\it entrée} (places précédentes),
	\item $Post:P \times T \rightarrow \mathbb{N}$ est l'application de {\it sortie} (places suivantes),
	\end{itemize}
\end{block}
\begin{block}{Réseau de Petri marqué $N = <\!R,M\!>$}
	\begin{itemize}
	\item $R$ est un réseau de Petri,
	\item $M : P \rightarrow \mathbb{N}$ est le marquage initial \\
		(distribution de jetons dans les places)
	\end{itemize}
\end{block}
\end{frame}

\begin{frame}{Définitions}
\begin{block}{Exemple}
\begin{columns}
	\begin{column}{.7\textwidth}
		\begin{itemize}
		\item $R = <\!P, T, Pre, Post\!>$\\
		\item $P=\{p_1,p_2,p_3\}$
		\item $T=\{a,b,c,d\}$
		\item $Post(p_1,a)=1$, $Pre(p_1,b)=1$, $Post(p_2,b)=1$
		\end{itemize}
	\end{column}
	\begin{column}{.3\textwidth}
		\includegraphics[width=2.8cm]{pre}
	\end{column}	
\end{columns}
\end{block}
\end{frame}

\begin{frame}{Graphe et notation matricielle}
\begin{block}{Réseau de Petri marqué $N=<\!R,M\!>$}
\begin{columns}
	\begin{column}{.5\textwidth}
		\includegraphics[width=5.5cm]{lececr}
	\end{column}
	\begin{column}{.6\textwidth}
		$$P=\{p_1,p_2,p_3\}, \qquad T=\{a,b,c,d\}$$
		$$Pre = \begin{pmatrix}
			0 & 1 & 0 & 0 \\
			1 & 0 & 3 & 0 \\
			0 & 0 & 0 & 1 
			\end{pmatrix}$$
		$$Post = \begin{pmatrix}
			1 & 0 & 0 & 0 \\
			0 & 1 & 0 & 3 \\
			0 & 0 & 1 & 0 
			\end{pmatrix}$$
		$${}^tM = \begin{pmatrix}0 & 3 & 0\end{pmatrix}$$
	\end{column}	
\end{columns}
\end{block}
\end{frame}

\begin{frame}{Règle de fonctionnement}
\begin{block}{Transition sensibilisée à partir de M}
	\begin{itemize}
	\item il y a un numéro suffisant de jetons dans les places d'entrée,
	\item $\forall p \in P, \;\;  M(p) \geq Pre(p, t)$
	\item $M \geq Pre(\, . \, ,t)$
	\end{itemize}
\end{block}
\begin{block}{Tir d'une transition à partir de M}
	\begin{itemize}
	\item $\forall p \in P, \;\; M'(p)=M(p)-Pre(p,t)+Post(p,t)$
	\item $M'=M-Pre(\, . \, ,t)+Post(\, . \, ,t)=M + C(\, . \, ,t)$
	\end{itemize}
\end{block}
\end{frame}

\begin{frame}{Règle de fonctionnement}
\begin{columns}
	\begin{column}{.6\linewidth}
		\begin{itemize}
		\item  Enlève $Pre(p,t)$ jetons de chaque place précédente $p$ (poids de l'arc d'entrée), et met $Post(p,t)$ jetons à chaque place de sortie $p$,
		\item  Représente le changement d'état dû à l'ocurrence de l'événement associé à $t$.
		\end{itemize}
	\end{column}
	\begin{column}{.4\linewidth}
		\begin{center}
		\includegraphics[width=\linewidth]{le}
		\end{center}
	\end{column}
\end{columns}
\end{frame}

\subsection{Structures}
\begin{frame}{Différentes interactions entre les processus}
\begin{block}{Séquence}
	\begin{center}
		\includegraphics[width=.6\linewidth]{fseq-inf}
	\end{center}
	\begin{itemize}
	\item séquence d'un processus de fabrication :
	   \begin{itemize}
	   \item $P_i$ : phase $i$ de l'opération sur la pièce, 
	   \item $t_i$ : passage d'une phase à une autre;
	   \end{itemize}
	\item portion de l'itinéraire d'un système de transport :     
	   \begin{itemize}
	   \item $P_i$ : chariot traverse la section $i$,
	   \item $t_i$ : passage d'un chariot d'une section  à une autre;
	   \end{itemize}
	\end{itemize}
%$\leadsto$ deux systèmes {\it différents}, mais avec un comportement analogue,
%possèdent le {\it même} modèle de RdP!
\end{block}
\end{frame}
 
\begin{frame}{Différentes interactions entre les processus}
\begin{block}{Branchement}
\begin{columns}
	\begin{column}{.3\linewidth}
		\begin{center}
			\includegraphics[width=\linewidth]{div}
		\end{center}
	\end{column}
	\begin{column}{.7\linewidth}
		\begin{itemize}
		\item à partir de l'activité $J_1$, deux activités sont crées ($J_2$ et $J_3$),
		\item $J_2$ et  $J_3$ évoluent de façon indépendante.
		\end{itemize} 
	\end{column}
\end{columns}
\end{block}
\end{frame}

\begin{frame}{Différentes interactions entre les processus}
\begin{block}{Jonction}
\begin{columns}
	\begin{column}{.3\linewidth}
		\begin{center}
			\includegraphics[width=\linewidth]{jun}
		\end{center}
	\end{column}
	\begin{column}{.7\linewidth}
		\begin{itemize}
		\item évolution indépendante de $t_1$ et $t_2$ (évolution asynchrone),
		\item synchronisme en $t_3$.
		\end{itemize}
	\end{column}
\end{columns}
\end{block}
\end{frame}

\begin{frame}{Différentes interactions entre les processus}
\begin{block}{Choix}
\begin{columns}
	\begin{column}{.3\linewidth}
		\begin{center}
			\includegraphics[width=\linewidth]{alt}
		\end{center}
	\end{column}
	\begin{column}{.7\linewidth}
		\begin{itemize}
		\item choix entre $t_2$ (seq. $P_2P_3$) et $t_3$ (seq. $P_4P_5$): seulement une peut être tirée;
		\item les 2 séquences exécuteront $P_6$.
		\end{itemize} 
	\end{column}
\end{columns}
\end{block}
\end{frame}

\begin{frame}{Différentes interactions entre les processus}
\begin{block}{Répétition}
\begin{columns}
	\begin{column}{.3\linewidth}
		\begin{center}
			\includegraphics[width=\linewidth]{rep}
		\end{center}
	\end{column}
	\begin{column}{.7\linewidth}
		\begin{itemize}
		\item choix entre $t_2$ e $t_3$,
		\item répéter la séq. $P_2P_3$ un certain nombre de fois avant de exécuter $P_4$.
		\end{itemize}
	\end{column}
\end{columns}
\end{block}
\end{frame}

\begin{frame}{Différentes interactions entre les processus}
\begin{block}{Allocation de ressources}
\begin{columns}
	\begin{column}{.3\linewidth}
		\begin{center}
			\includegraphics[width=\linewidth]{frecurso}
		\end{center}
	\end{column}
	\begin{column}{.7\linewidth}
		\begin{itemize}
		\item un même chariot doit servir différentes machines,
		\item un opérateur doit exécuter  différentes activités (une à la fois).
		\end{itemize}  
	\end{column}
\end{columns}
\end{block}
\end{frame}

\begin{frame}{Exemple : Système par lot}
\begin{itemize}
\item peut produire deux produits ($Pr_1$ et $Pr_2$), utilisant 2 réacteurs
($R_1$ e $R_2$) de fa\c{c}on concurrente,
\item produit $Pr_1$: est produit par $R_1$ ou $R_2$; doit être, au préalable,
stocké dans le {\it buffer} $B_1$ ou $B_2$ (respectivement).
\item produit $Pr_2$: est produit par le réacteur $R_2$.
\end{itemize}
\begin{center}
\includegraphics[width=.4\linewidth]{reacteur}
%\includegraphics[width=.4\linewidth]{freator}
\end{center}
\end{frame}
 
\subsection{Modèle dynamique}
\begin{frame}{Conflit et parallélisme}
\begin{itemize}
\item \structure{Conflit structurel :} ssi  $t_1$ et $t_2$ ont au moins une place d'entrée en commun
	$$\exists p \in P, \quad Pre(p,t_1) \, Pre(p,t_2) \neq 0$$
\item \structure{Conflit effectif :} ssi $t_1$ et $t_2$ sont en conflit structurel et sont sensibilisées par le marquage $M$
	$$M \geq Pre(.,t_1) \; \mbox{et} \; M \geq  Pre(.,t_2)$$
\item \structure{Parallélisme structurel :} si $t_1$ et $t_2$ ne possèdent pas de place d'entrée en commun
	$$\forall p \in P \quad Pre(p,t_1) \, Pre(p,t_2) = 0  \;
	\mbox{ou} \; Pre(.,t_1)^T \times Pre(.,t_2) = 0$$
\item \structure{Parallélisme effectif :} $t_1$ et $t_2$ sont parallèles structurellement et
$$M \geq Pre(.,t_1) \; \mbox{e} \; M \geq Pre(.,t_2)$$
\end{itemize}
%$$Pre(.,b) = \begin{pmatrix} 1 \\ 0 \\ 0 \end{pmatrix}$$
%$$Pre(.,d) = \begin{pmatrix} 0 \\ 0 \\ 1 \end{pmatrix}$$
\end{frame}

\begin{frame}{Séquence de tir}
$$\begin{array}{ccccccc}
\begin{pmatrix} 0\\ 3\\ 0 \end{pmatrix} & \stackrel{a}{\longrightarrow} &
\begin{pmatrix} 1\\ 2\\ 0 \end{pmatrix} & \stackrel{a}{\longrightarrow} &
\begin{pmatrix} 2\\ 1\\ 0 \end{pmatrix} & \stackrel{b}{\longrightarrow} &
\begin{pmatrix} 1\\ 2\\ 0 \end{pmatrix}\\
M_0 & & M_1 & & M_2 & & M_1
\end{array}$$
\begin{itemize}
\item $M'$ accessible à  partir de $M$: $M \stackrel{t}{\longrightarrow} M'$
\item $M_0 \stackrel{a}{\longrightarrow} M_1$, $M_1 \stackrel{a}{\longrightarrow} M_2$, $M_2 \stackrel{b}{\longrightarrow} M_1$, 
\item si $M \stackrel{t_1}{\longrightarrow} M'$, et $M' \stackrel{t_2}{\longrightarrow} M''$, on a $s=t_1t_2$ et $M \stackrel{t_1t_2}{\longrightarrow} M''$
\item dans l'exemple, $M_0 \stackrel{s}{\longrightarrow} M_1$, avec $s=aab$, $s$ est dite séquence de tir
\begin{eqnarray*}
{\bf s} : &T& \rightarrow \mathbb{N}\\
		  &t& \mapsto \mbox{ nombre d'occurrences de $t$ dans $s$}
\end{eqnarray*}
\end{itemize}
\end{frame}

\begin{frame}{Séquence de tir}
\begin{itemize}
\item \structure{Équation fondamentale :} \fbox{$M'=M+C{\bf s}$}
\item Etant donné $M$ et une sequence $s$, existe-t-il $M'$ t.q. $M \stackrel{s}{\longrightarrow} M'$~?
\item Etant donné $M$ et $M'$, existe-t-il $s$ t.q. $M \stackrel{s}{\longrightarrow} M'$~?
\end{itemize}
\end{frame}
 
%\subsection{Autres représentations}
%\begin{frame}{Système de règles}
%\begin{itemize}
%\item une base de faits, représentant la connaissance disponible sur le système,
%\item une base de règles, qui permet de déduire de nouveaux faits,
%\item un {\em moteur d'inférence}, qui permet de
%réaliser de nouvelles déductions, appliquant les règles aux faits.
%\end{itemize}
%\end{frame}

%\begin{frame}{Grammaire}
%\begin{itemize}
%\item un alphabet $\Pi$ dont les symboles sont les places $p \in P$
%\item un ensemble $Q$ de règles de réécriture :
%$$t: \mu(Pre(.,t)) \rightarrow \mu(Post(.,t))$$
%\item $\mu: \mathcal{M} \rightarrow {\Pi}^*$, $\mathcal{M}$ l'ensemble de tous les marquages et ${\Pi}^*$ l'ensemble des séquences finies (dont la séquence vide $\lambda$)
%\item transition sensibilisée : 
%$$\mu(Pre(.,t)) \subseteq \mu(M)$$
%\item tir de transition : 
%$$\mu(M')=\mu(M) \cup \mu(Post(.,t)) \setminus \mu(Pre(.,t))$$
%\end{itemize}
%\end{frame}

%%%%% PROPRIETES %%%%%
\section{Analyse de propriétés}
\begin{frame}
\tableofcontents[hideothersubsections]
\end{frame}

\subsection{Propriétés comportementales}
\begin{frame}{Réseau borné}
\begin{itemize}
\item Place \structure{k-bornée} : le nombre maximal de jetons de la place, pour tout marquage accessible, est plus petit que $k$
$$\forall M' \in \mathcal{A}(\mathcal{R}, M_0), \quad M'(p) \leq k$$
\item Si $k = 1$, la place est dite \structure{binaire},
\item Un réseau marqué est \structure{k-borné} ssi toutes ses places le sont
\item Un réseau marqué est \structure{binaire} ssi toutes ses places le sont
\end{itemize}
\begin{center}
\includegraphics[width=.25\linewidth]{nonbor}
\end{center}
\end{frame}
 
\begin{frame}{Réseau vivant}
\begin{itemize}
\item Transition \structure{quasi-vivante} : 
$$\exists s \, / \; M_0 \stackrel{s}{\longrightarrow} M \quad \mbox{et} \quad 
M \stackrel{t}{\longrightarrow}$$
\item Transition \structure{vivante} :
$$\forall M \in {\cal A(R},M_0), \; \exists s \, / \; M \stackrel{st}{\longrightarrow}$$
\item Réseau \structure{vivant} ssi toutes ses transitions sont vivantes
\end{itemize}
\begin{center}
\includegraphics[width=.35\linewidth]{qviva-n}
\end{center}
\end{frame}
  
\begin{frame}{Réseau réinitialisable}
\begin{itemize}
\item Réseau marqué \structure{réinitialisable} s'il est possible de revenir au marquage initial à  partir de n'importe quel marquage :
$$\forall M \in {\cal A(R}, M_0), \quad \exists s \, / \; M \stackrel{s}{\longrightarrow} M_0$$
\end{itemize}
\begin{center}
\includegraphics[width=.7\linewidth]{nonrei-g}
\end{center}
\end{frame}

\subsection{Propriétés structurelles}
\begin{frame}{Composantes conservatives}
\begin{itemize}
\item circuit formé par $p_1$, $p_2$, $a$, $b$: $M(p_1) + M(p_2)$ est constant
	\begin{itemize}
	\item $M_0 = {}^t\begin{pmatrix}1 & 0 & 3 & 0 & 1\end{pmatrix}$
	\item $M_0 \stackrel{a}{\longrightarrow} 
		M'= {}^t\begin{pmatrix}0 & 1 & 2 & 0 & 1\end{pmatrix}$
	\item $M' \stackrel{b}{\longrightarrow} M'' = M_0$
	\end{itemize}
\end{itemize}
\begin{center}
	\includegraphics[width=.4\linewidth]{inv}
\end{center}
\begin{itemize}
\item Marquage obtenu après une séquence de tir : $M' = M + Cs$
\item Composante conservative : $f \, / \; {}^tf \, C = 0$
\item dans l'exemple :
$$\scriptstyle
{}^tf^1 = \begin{pmatrix}1&1&0&0&0\end{pmatrix}, \;
{}^tf^2 = \begin{pmatrix}0&1&1&3&0\end{pmatrix}, \;
{}^tf^3 = \begin{pmatrix}0&0&0&1&1\end{pmatrix}$$
\end{itemize}
\end{frame}

\begin{frame}{Invariants de place}
\begin{itemize}
\item Invariant de place = composante conservative + marquage
\item ${}^tf \, C = 0 \Rightarrow {}^tf \, M = {}^tf \, M_0 \quad \forall M \in {\cal A(R}, M_0)$
\item $M(p_1) + M(p_2) = M_0(p_1) + M_0(p_2)=1$
\item $M(p_2) + M(p_3) + 3.M(p_4) = 3$  
\item $M(p_4) + M(p_5) =1$  
\end{itemize}
{\bf Remarque} 
\begin{itemize}
\item composante conservative : dépend seulement de la structure !
\item invariant de place : dépend de la structure \structure{et} du marquage
\end{itemize}
\end{frame}
  
\begin{frame}{Composantes répétitives stationnaires}
\begin{itemize}
\item Sous-réseau formé par $c$ et $d$, et places $p_3$, $p_4$ et $p_5$ : le tir de $s = cd$ à partir de $M_0$ ramène au même marquage
\item Transitions $c$ et $d$ forment une \structure{composante répétitive stationnaire}
\item $M' = M \Rightarrow C \, s = 0 \quad s \mbox{ composante répétitive}$
\item Dans l'exemple :
$$s^1 = {}^t\begin{pmatrix}1&1&0&0\end{pmatrix}, \;
s^2 = {}^t\begin{pmatrix}0&0&1&1\end{pmatrix}$$
\end{itemize}
\begin{center}
	\includegraphics[width=.4\linewidth]{inv}
\end{center}
\end{frame}
  
\begin{frame}{Invariants de transition}
\begin{itemize}
\item Séquences $s_i$ obtenues à partir du vecteur $s$
\item Pour $s_1$ on peut avoir les invariants $s_{11} = ab$ et $s_{12} = ba$
\item Il faut calculer $M \stackrel{ab}{\longrightarrow}$ et $M \stackrel{ba}{\longrightarrow}$ pour vérifier !
\end{itemize}  
{\bf Remarque} 
\begin{itemize}
\item Composante répétitive : dépend seulement de la structure !
\item Invariant de transition : dépend de la structure \structure{et} du  marquage
\end{itemize}
\end{frame}

%\begin{frame}{Exemple : Système par lot}
%\begin{columns}
%	\begin{column}{.4\linewidth}
%	\begin{center}
%		\includegraphics[width=\linewidth]{rea1}
%	\end{center}
%	\end{column}		
%	\begin{column}{.66\linewidth}
%		\begin{itemize}
%		\item $f^1 = {}^t\begin{pmatrix}1&1&1&1&1&0&0&0&0\end{pmatrix}$~:
%			fabrication de $Pr_1$
%		\item $f^2 = {}^t\begin{pmatrix}0&0&0&0&0&1&1&0&0\end{pmatrix}$~:
%			fabrication de $Pr_1$
%		\item $f^3 = {}^t\begin{pmatrix}0&0&1&0&0&0&0&1&0\end{pmatrix}$~:
%			état du réacteur $R_1$
%		\item $f^4 = {}^t\begin{pmatrix}0&0&0&0&1&0&1&0&1\end{pmatrix}$~:
%			état du réacteur $R_2$
%		\end{itemize}
%	\end{column}
%\end{columns}
%\end{frame}

%\begin{frame}{Exemple : Système par lot}
%\begin{columns}
%	\begin{column}{.4\linewidth}
%	\begin{center}
%		\includegraphics[width=\linewidth]{rea1}
%	\end{center}
%	\end{column}		
%	\begin{column}{.65\linewidth}
%		\begin{itemize}
%		\item $M(p_1)+M(p_2)+M(p_3)+M(p_4)+M(p_5)=1$~: un seul état pour $Pr_1$ (attente, $B_1$, $B_2$, $R_1$, $R_2$)
%		\item $M(p_6)+M(p_7)=1$~: un seul état pour $Pr_2$ (attente ou $R_2$)
%		\item $M(p_3)+M(p_8)=1$~: $R_1$ libre ou prod. $Pr_1$
%		\item $M(p_5)+M(p_7)+M(p_9)=1$~: $R_2$ libre ou prod. $Pr_1$ ou $Pr_2$
%		\end{itemize} 
%	\end{column}
%\end{columns}
%\end{frame}
 
%\begin{frame}{Exemple : Système par lot}
%\begin{columns}
%	\begin{column}{.4\linewidth}
%	\begin{center}
%		\includegraphics[width=\linewidth]{rea1}
%	\end{center}
%	\end{column}		
%	\begin{column}{.65\linewidth}
%		\begin{itemize}
%		\item $s^1 = {}^t\begin{pmatrix}1&1&1&0&0&0&0&0\end{pmatrix}$~:
%		$Pr_1$ passe par le {\em buffer} et est prod. par $R_1$ ; un nouveau cycle peut recommencer ;
%		\item $s^2 = {}^t\begin{pmatrix}0&0&0&1&1&1&0&0\end{pmatrix}$~:
%		même comportement pour $Pr_2$ et $R_2$ ;
%		\item $s^3 = {}^t\begin{pmatrix}0&0&0&0&0&0&1&1\end{pmatrix}$~:
%		début de prod. de $Pr_2$ et fin
%		\end{itemize} 
%	\end{column}
%\end{columns}
%\end{frame}

%\begin{frame}{Exemple : Système par lot}
%\begin{columns}
%	\begin{column}{.4\linewidth}
%	\begin{center}
%		\includegraphics[width=\linewidth]{rea1}
%	\end{center}
%	\end{column}		
%	\begin{column}{.65\linewidth}
%		Seules séquences effectivement réalisables :
%		\begin{itemize}
%		\item $s_1 = t_a t_b t_c$
%		\item $s_2 = t_d t_e t_f$
%		\item $s_3 = t_g t_h$
%		\end{itemize} 
%	\end{column}
%\end{columns}
%\end{frame}

\subsection{Analyse}
\begin{frame}{Analyse des propriétés}
\begin{description}
\item[Analyse par énumération des marquages] le graphe des marquages accessibles est calculé, vérifiant si le réseau est borné, vivant et réinitialisable.
\item[Analyse structurelle] calcul des composantes conservatives et répétitives stationnaires et des invariants correspondants ; ne permet pas toujours d'avoir une réponse, mais dans certains cas, permet d'obtenir une réponse simples et rapide des propriétés du réseau.
\item[Analyse par réduction] si le réseau est trop grand ou non borné, on peut réduire la taille du réseau, en utilisant certaines règles de réduction.
\end{description}
\end{frame}
 
\begin{frame}{Analyse par énumération des marquages}
\begin{block}{Arbre de couverture}
\begin{itemize}
\item On part du marquage initial $M_0$,
\item On crée une branche pour chaque transition sensibilisée par $M_0$,
\item La construction d'une branche est interrompue quand on rencontre un marquage
	\begin{itemize}
	\item déjà calculé,
	\item strictement supérieur à un marquage {\em de la branche qui est en train d'être explorée}.
	\end{itemize}
\end{itemize}  
Si le réseau est non borné, on introduit le symbole $\omega$ pour rendre l'arbre fini.
\end{block}
\end{frame}  
 
\begin{frame}{Analyse par énumération des marquages}
\begin{block}{Recherche des propriétés sur ${\cal A(R},M)$}
\begin{itemize}
\item Réseau \structure{$k$-borné} $\Leftrightarrow {\cal A(R},M)$ borné
\item Réseau \structure{réinitialisable} $\Leftrightarrow {\cal A(R},M)$ fortement connexe
$$\forall M_i, M_j \in {\cal A(R},M), \; \exists s \, / \; M_i \stackrel{s}{\rightarrow} M_j$$
\item Réseau \structure{vivant} $\Leftrightarrow {\cal A(R},M)$ fortement connexe et chaque transition étiquette au moins un arc
$$\forall t \in T, \exists M_i, M_j \in {\cal A(R},M), \; / \; M_i
\stackrel{t}{\rightarrow} M_j$$
\end{itemize}  
\end{block}
\end{frame}  

%\begin{frame}{Analyse par énumération des marquages}
%\begin{center}
%\includegraphics[width=.8\linewidth]{qviva-g}
%\end{center}
%\end{frame}  
 
\begin{frame}{Analyse structurelle}
\begin{block}{Composantes conservatives}
\begin{itemize}
\item Toute place qui \structure{appartient} à une composante conservative  est \structure{bornée}
\item Une place $p$ qui n'appartient à aucune composante conservative ($f(p)=0$) peut être bornée
\item Une place \structure{non bornée} n'appartient à \structure{aucune} composante conservative
\end{itemize}
Un réseau de Petri pour lequel il existe une couverture de composantes 
conservatives ($f>0$) est \structure{k-borné}, {\it peu importe son marquage initial.}
\end{block}
\end{frame}  
 
\begin{frame}{Analyse structurelle}
\begin{block}{Invariants de place}
${}^tf \, M = {}^tf \, M_0$ permet de calculer une limite pour \structure{chaque place} $p$
$$f(p) M(p) \leq {}^tf \, M_0, \qquad M(p) \leq \frac{{}^tf \, M_0}{f(p)}$$
\end{block}
\end{frame}  

\begin{frame}{Analyse structurelle}
\begin{block}{Composantes répétitives}
Réseau de Petri répétitif : il existe une couverture de composantes répétitives ($s>0$)
	\begin{itemize}
	\item un réseau de Petri \structure{borné et vivant} est \structure{répétitif}
	\item un réseau \structure{non répétitif} ($\exists t, \, s(t)=0$) est \structure{non vivant ou non borné}
	\end{itemize}
\end{block}
\end{frame}  
   
\begin{frame}{Exemple}
\begin{center}
	\includegraphics[width=3cm]{exemplea} 
\end{center}
\end{frame}

\begin{frame}{Exemple : Système par lot}
\begin{center}
	\includegraphics[width=.8\linewidth]{rea1}
\end{center}
\end{frame}

\subsection{Tina}
\begin{frame}
\begin{small}
\begin{itemize}
\item Ouvrir une fenêtre de commandes en ligne ; taper "nd" (NetDraw)
\item Faire Help, Setup (5eme ascenceur): indiquer le nombre de boutons de la souris
\item Faire Help, Help : comment créer les  places, les transitions, les arcs, comment changer les propriétés d'une place (marquage, label), d'un arc (poids), d'une transition (label). Comment déplacer, effacer les éléments...
\item Editer le RdP des lecteurs écrivains
\item Générer le graphe d'accessibilité (tools/reachability analysis/ cocher "marking graph", utiliser comme sortie "lts(.aut)".
\item Dessiner ce graphe : click droit, "open file in nd" ; edit/draw ; déplacer les noeuds pour que le graphe soit lisible, et "séparer" les doubles flèches
\item Générer le graphe d'accessibilité (tools/reachability analysis/ cocher "marking graph", utiliser comme sortie "verbose".
\end{itemize}
\end{small}
\end{frame}    

%%%%% COMPOSITION %%%%%
\section{Composition}
\begin{frame}
\tableofcontents[hideothersubsections]
\end{frame}

\subsection{Composition}
\begin{frame}{Caractéristiques de la représentation par RdP}
\begin{itemize}
\item \structure{Modularité} : est-il possible de décomposer un système complexe?
\item \structure{Composition} : si les modules ont les bonnes propriétés, la composition de ces modules garde-t-elle les bonnes propriétés? Ou est-il nécessaire d'analyser le système globale (composé)?
\item \structure{Calculabilité} : existe-t-il des algorithmes pour l'analyse?
\end{itemize}
\begin{block}{Bloc bien-formé}
	un réseau de Petri avec :
	\begin{itemize}
	\item une transition d'entrée $t_e$ et une transition de sortie $t_s$,
	\item réseau borné, vivant et réinitialisable si l'on rajoute une place $p$ tel que $Pre(p,t_e)=Post(p,t_s)=1$. \\
	Ex: séquence, if-then-else, do-while, fork-join.
	\end{itemize}
\end{block}
\end{frame}

\begin{frame}{Raffinement}
Processus d'abstraction fait en deux temps :
\begin{itemize}
\item modélisation d'une  première ébauche ({\it vision abstraite du système global}), avec des transitions {\it abrégées} (associées à des tâches complexes),
\item à partir de ce RdP, remplacer les transitions {\it abrégées} par des \structure{blocs bien-formés} représentant une vision détaillée de ces tâches complexes
\item conception {\it top-down}
\item analyse : si le réseau abstrait est un bloc bien formé, et les transitions sont représentées par des blocs bien formés, alors le réseau global est bien formé.
\end{itemize} 
\end{frame}

\begin{frame}{Composition}
Processus "à objets" :
\begin{itemize}
\item modélisation \structure{détaillée} de deux objets dès le départ
\item construction du réseau global à partir de la composition de ces objets
	\begin{itemize}
	\item Composition \structure{asynchrone} (fusion des places)
	\item Composition \structure{synchrone} (fusion des transitions)
	\end{itemize}
\item conception {\it bottom-up}
\item analyse : le réseau global composé ne conserve pas forcement les propriétés de chaque bloc
\end{itemize}
\end{frame}
  
\begin{frame}{Composition synchrone}
Fusion des transitions $t_1$ et $t_3$ ($t_{13}$) et de $t_2$ et $t_4$ ($t_{24}$)
\begin{center}
\includegraphics[width=.8\linewidth]{compoa}
\end{center}
\end{frame}

\begin{frame}{Composition asynchrone}
Fusion des places $p_2$ et $p_7$ ($p_{27}$)
\begin{columns}
\begin{column}{.5\linewidth}
	\centering
	\includegraphics[width=\linewidth]{exempleb}
\end{column}
\begin{column}{.5\linewidth}
	\centering
	\includegraphics[width=\linewidth]{exemplec}
\end{column}
\end{columns}
\end{frame}

\begin{frame}{Communication par place}
Deux processus $A$ et $B$ exécutant chacun une opération doivent se communiquer :
\begin{itemize}
\item $A$ ne peut commencer qu'après la fin de $B$ ;
\item $B$ doit attendre que $A$ commence pour pourvoir commencer.
\end{itemize}
\begin{columns}
\column{.27\textwidth}
	\includegraphics[width=3.2cm]{comp2Seul_ndr}\\
	processus indépendants \pause
\column{.35\textwidth}
	\includegraphics[width=4.4cm]{compSep_ndr} \\
	modèle local\pause
\column{.3\textwidth}
	\includegraphics[width=3.5cm]{comp3_ndr}\\
	modèle global
\end{columns}
\end{frame}

%\begin{frame}{Retour aux RdP : composition synchrone}
%{\footnotesize À partir des RdP représentant les philosophes (P et Q) et les baguettes (1 et 2), construire le RdP du système global (partage des baguettes par les 2 philosophes) en utilisant la composition synchrone (fusion des transitions avec le même nom ou correspondant au même événement)}
%\begin{columns}
%\column{.5\textwidth}  
%	{\scriptsize \hspace{1cm}P \hspace{3cm}Q~~~}\\
%	\includegraphics[width=3cm]{philP_ndr}
%	\includegraphics[width=3.2cm]{philQ_ndr}  % PQ
%\pause
%\column{.5\textwidth}
%	{\scriptsize \hspace{1cm}b1 \hspace{2.2cm}b2~~~}\\
%	\includegraphics[width=2.5cm]{baguette1_ndr} 
%	\includegraphics[width=2.5cm]{baguette2_ndr} \\
%	{\scriptsize Fusion de $t$ de $R_1$ avec  $t$ de $R_2$ : $t_{12}$\\
%	\begin{itemize}
%	\item soit $p \in R_1$ et $q \in R_2$ ; $r \in R_{12}$
%	\item Pre$(r,t_{12})$=Pre$(p,t_1)$ et Pre$(q,t_2)$
%	\item Post$(r,t_{12})$=Post$(p,t_1)$ et Post$(q,t_2)$
%	\end{itemize}
%	}
%\end{columns}
%\end{frame}

%\begin{frame}{Retour aux RdP : composition synchrone}
%\begin{columns}
%\column{.6\textwidth}
%	{\footnotesize RdP global PQb1b2 obtenu par composition synchrone de P, Q, b1 et b2}
%	\includegraphics[width=7cm]{PQb1b2_ndr}
%\column{.4\textwidth}
%	{\footnotesize Graphe de marquage du RdP PQb1b2 = automate pg~60}\\%\pageref{phil_aut}
%	\includegraphics[width=4cm]{PQb1b2_aut}
%\end{columns}
%\end{frame}

%%% TEMPOREL %%%
\section{Réseau de Petri et le temps}
\begin{frame}
\tableofcontents[hideothersubsections]
\end{frame}

\subsection{Introduction}
\begin{frame}{Réseau de Petri et le temps}
Plusieurs extensions de RdP pour prendre en compte le temps :
\begin{itemize}
\item temps associé aux arcs,
\item temps associé aux places,
\item temps associé aux transitions (RdP temporel, RdP temporisé)
\end{itemize}
\begin{block}{Réseaux de Petri temporels (Merlin, 1974)}
	\begin{center}
	\includegraphics[width=.3\linewidth]{rdp_s_forte}
	\end{center}
\end{block}
\end{frame}
 
\subsection{RdP $t$-temporels}
\begin{frame}{Réseaux de Petri t-temporels}
Un réseau de Petri t-temporel $<N, M_0, I>$ est défini par~:
\begin{itemize}
\item un réseau de Petri \mbox{$N = <P,T,Pre, Post>$},
\item un marquage initial $M_0$,
\item une fonction intervalle statique $I$ :
	$$I: T \rightarrow (Q^+ \cup 0) \times (Q^+ \cup \infty)$$
\end{itemize}
\vspace{.2cm}
Protocole unidirectionnel de transfert de données :
\begin{center}
\includegraphics[width=.8\linewidth]{rdp_exo}
\end{center}
\end{frame}

\begin{frame}{Réseaux de Petri t-temporels}
Intervalle temporel $I(t_i)=[a_i,b_i]$ : dates de tir possibles de $t_i$ à partir de sa date de
sensibilisation.
\begin{center}
\includegraphics[width=.8\linewidth]{rdp_exo}
\end{center}
\begin{itemize}
\item date de \structure{sensibilisation} d'une transition $t_i$
\item date de \structure{début} et de \structure{fin} de l'intervalle de tir,
\item date de \structure{franchissement} effectif de $t_i$.
\end{itemize}
\end{frame}

\begin{frame}{Sémantique}
\begin{itemize}
\item si plusieurs transitions franchissables $I(t_i)=[a_i,b_i]$: franchir l'une d'elles avant la
	\structure{fin} de l'intervalle de tir des autres transitions.
\item tir de $t_1$ avant $t_2$ ($b_2=\structure{4}$) $\rightarrow$ donc, $t_1 \, [1 \, 3]$.
\item tir $t_2$ avant $t_1$ ($b_1=\structure{3}$) $\rightarrow$ donc, $t_2$ $[2 \, 3]$
\end{itemize}
\begin{center}
	\includegraphics[width=.4\linewidth]{rdp_s_forte}
\end{center}
\end{frame}

%------------------
%\subsection{Réseaux de contraintes temporelles}
%\begin{frame}
%\frametitle{Réseaux de contraintes temporelles}

%Représente les contraintes temporelles à un moment donné

%\vspace{1ex}
%\begin{columns}

%\column{.4\textwidth}
%\centerline{\includegraphics[width=3cm]{figCharles/rdp_stn}}

%\column{.3\textwidth}
%\only<1>{\centerline{\includegraphics[width=2.7cm]{figCharles/stn0_nodes}}}
%\only<2>{\centerline{\includegraphics[width=2.7cm]{figCharles/stn0}}}%avec arcs

%\only<3,4>{\centerline{\includegraphics[width=2.7cm]{figCharles/stn1}}}%avec [0,inf)

%\column{.3\textwidth}
%\only<4>{\centerline{\includegraphics[width=2.7cm]{figCharles/stn4}}}

%\end{columns}

%\begin{itemize}
%\item  Ensemble fini de variables : $X$ = \{$x_0$, $x_1$, $x_2$\}
%    \begin{itemize}
%    \item \{$x_1$, $x_2$\} $\backsim$ transitions sensibilisées : $t_1$, $t_2$

%    \item $x_0$ $\backsim$ date \textcolor[rgb]{0.00,0.00,1.00}{sensibilisation}
%    de $t_1$ et $t_2$ (arrivée de jetons)
%    \end{itemize}

%  \only<2,3,4>{\item Ensemble fini de contraintes $C$ :
%      \begin{itemize}
%      \item ($x_0$, $x_1$) = [1~2]; ($x_0$, $x_2$) = [3~4] $\backsim$
%      \textcolor[rgb]{0.00,0.00,1.00}{intervalle de tir} de $t_1$, $t_2$
%      \only<3,4>{\item ($x_1$, $x_2$) = [0~$\infty$) $\backsim$
%      franchir $t_2$ après $t_1$}
%      \end{itemize}}%only item

%\only<4>{\item Algorithme de Floyd-Warshall  (plus court chemin):
%STN \emph{minimal}.\\
% {\small $\blacktriangleright$  \textcolor[rgb]{0.00,0.00,1.00}{franchissement} de la transition $t_2$ :
% [1~3] après $t_1$}}
%\end{itemize}

%\end{frame}

\subsection{Graphe de classes}
\begin{frame}{Graphe de marquage classique}
\begin{columns}
\column{.45\textwidth}
	\centerline{\includegraphics[height=4cm]{rdp_sanst}} \vspace{1ex}
	\centerline{Réseau de Petri atemporel}
\column{.05\textwidth} $\stackrel{analyse}{\longrightarrow}$
\column{.50\textwidth}
	\centerline{\includegraphics[height=4cm]{graph_m}} \vspace{1ex}
	\centerline{Graphe de marquage}
\end{columns}
\end{frame}

\begin{frame}{Graphe de classes}
\begin{columns}
\column{.4\textwidth}
	\begin{center}
		\includegraphics[height=4cm]{rdp_temps_ch}\\
		Réseau de Petri \structure{temporel}
	\end{center}
\column{.6\textwidth}
	\begin{itemize}
	\item Prise en compte du temps :
		\begin{itemize}
		\item Nombre infini d'états (marquage + temps)
		\item Nombre infini de séquences
		\end{itemize}
	\item Il faut :
		\begin{itemize}
		\item Regrouper les états en un nombre fini de classes : \emph{oublier} une partie du passé.
		\item Classe ${\cal C}$ : donne les  intervalles de tir et les contraintes temporelles que doivent vérifier les transitions vis-à-vis des franchissements passés.
		\end{itemize}
	\end{itemize}
\end{columns}
\end{frame}

\begin{frame}{Graphe de classes}
\begin{columns}
\column{.4\textwidth}
	\begin{center}
		\includegraphics[height=4cm]{rdp_temps_ch}\\
		Réseau de Petri \structure{temporel}
	\end{center}
\column{.6\textwidth}
	\begin{itemize}
	\item \textbf{État}: \{$M$, $I(t)$\}
		\begin{itemize}
		\item $M$: Marquage
		\item $I(t)$: Fonction temporelle
		\end{itemize}
	\item \textbf{Classe d'états}:\\
		\small composée par tous les états que sont atteignables par une séquence de tir.
	\item Plusieurs définitions de classe, selon le type de propriétés à prouver :
		\begin{itemize}
		\item mode Linear,  {\footnotesize (B.Berthomieu)} % (LTL properties)%
		\item mode Arborescent,  {\footnotesize (B.Berthomieu)} % (CTL properties)%
		\item mode C,  {\footnotesize (J. Cardoso, R. Valette)} 
		\end{itemize}
	\end{itemize}
\end{columns}
\end{frame}

\begin{frame}{Mode linéaire}% {\footnotesize (B.Berthomieu)}
\begin{columns}
\column{.45\textwidth}
	\begin{center}
		\only<1-2>{\includegraphics[height=4cm]{rdp_temps_ch}}%
		\only<3>{\includegraphics[height=4cm]{rdp_c1}}%
		\only<4,5>{\includegraphics[height=4cm]{rdp_c2}}%
		\only<6>{\includegraphics[height=4cm]{rdp_c3}}%
		~\\Réseau de Petri \structure{temporel}
	\end{center}
\column{.55\textwidth} 
	\begin{center}
		\only<1-2>{\includegraphics[width=.7\linewidth]{horloge_che_a}\\}%
		\only<3>{\includegraphics[width=.7\linewidth]{horloge_che1_a}\\}%
		\only<4>{\includegraphics[width=.7\linewidth]{horloge_che2_a}\\}%
		\only<5>{\includegraphics[width=.7\linewidth]{horloge_che3_a}\\}%
		\only<6>{\includegraphics[width=.7\linewidth]{horloge_che4_a}\\}%
		%
		\only<1-2>{\includegraphics[width=.5\linewidth]{graph_wd}}%
		\only<3-6>{\includegraphics[width=.5\linewidth]{graph_wd2}}%
		%
		\only<2>{\\\structure{Tous ces états sont-ils atteignables!?}}%
		\only<3-6>{\\Si $t2$ est franchie au temps \structure{3}, $t3$ est-elle encore franchissable?}%
		\only<6>{ \alert{Non !}}
	\end{center}
\end{columns}
\end{frame}

\begin{frame}{Mode arborescent}% {\footnotesize (B.Berthomieu)}
\begin{columns}
\column{.5\textwidth}
	\centerline{\includegraphics[width=.8\linewidth]{graph_wd}}
	\begin{itemize}
	\item Problème branchement~:  distinguer les états dans le futur.
	\end{itemize}
\column{.5\textwidth}
	\centerline{\includegraphics[width=.8\linewidth]{graph_a}}
	\begin{itemize}
	\item Problème branchement : résolu
	\item Mais encore problème de chemin : distinguer les états dans le passé.
	\end{itemize}
\end{columns}
\end{frame}

\begin{frame}{Mode C}%Proposition : g
\begin{columns}
\column{.5\textwidth}
	\centerline{\includegraphics[width=.8\linewidth]{graph_a}}
	\begin{itemize}
	\item Problème branchement : résolu
	\item Mais encore problème de chemin : distinguer les états dans le passé.
	\end{itemize}
\column{.5\textwidth}
	\centerline{\includegraphics[width=.8\linewidth]{graph_c}}
	\begin{itemize}
	\item Problème branchement : résolu
	\item Problème de chemin : résolu
	\end{itemize}
\end{columns}
\end{frame}

\begin{frame}{Mode linéaire}
Graphe de classes :
\begin{itemize}
\item noeuds (classes $C_i$) :
	\begin{itemize}
	\item états avec le même marquage,
	\item domaine temporel (union des domaines temporels des états) :
		\begin{itemize}
		\item  intervalle de temps des transitions sensibilisées
		\item  contraintes temporelles entre couples de transitions sensibilisées (mémoire temporelle depuis la classe où elles étaient sensibilisées) ;
		\end{itemize}
	\end{itemize}
\item arcs $(C_i,C_j)$ : intervalle de tir de $t$, avec $C_i \stackrel{t}{\rightarrow}C_j$
\end{itemize}
\includegraphics[height=3cm]{rdp_temps_ch}
\end{frame}

\begin{frame}{Protocole unidirectionnel de transfert de données}
\centerline{\includegraphics[width=.7\linewidth]{rdp_exo}}
RdP \structure{sans} le temps : Graphe de couverture\\
\begin{columns}
\begin{column}{.4\linewidth}
	\includegraphics[width=1.2\linewidth]{ExIfac_gCk}
\end{column}
\begin{column}{.6\linewidth}
	\begin{description}
	\item[0] : p1 p5 p6\\
	\item[1] : p1 p2*w p5 p6\\
	\item[2] : p1 p2*w p3*w p5 p6\\
	\item[3] : p1 p2*w p3*w p4 p6\\
	\end{description}
\end{column}
\end{columns}
\end{frame} 

\begin{frame}{Protocole unidirectionnel de transfert de données}
Les classes (marquage + domaine temporel) :
\begin{center}
	\includegraphics[width=.6\linewidth]{w}
\end{center}
\includegraphics[width=.4\linewidth]{clas1}\\
\includegraphics[width=.6\linewidth]{cla2}
\end{frame}

\end{document}
